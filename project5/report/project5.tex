\documentclass[]{article}
\usepackage{amssymb}
\usepackage{amsmath}
\usepackage[utf8]{inputenc}
\usepackage{graphicx}
\usepackage{booktabs}
\usepackage{listings}
\usepackage{color}
\usepackage{tabularx}
\usepackage{hyperref}

\definecolor{dkgreen}{rgb}{0,0.6,0}
\definecolor{gray}{rgb}{0.5,0.5,0.5}
\definecolor{mauve}{rgb}{0.58,0,0.82}

\lstset{frame=tb,
	%language=C++,
	aboveskip=3mm,
	belowskip=3mm,
	showstringspaces=false,
	columns=flexible,
	basicstyle={\small\ttfamily},
	numbers=none,
	numberstyle=\tiny\color{gray},
	keywordstyle=\color{blue},
	commentstyle=\color{dkgreen},
	stringstyle=\color{mauve},
	breaklines=false,
	breakatwhitespace=true,
	tabsize=2
}


\title{FYS4150 H20 - Project 5:\\Disease Modelling with Runge-Kutta and\\Monte Carlo Simulations}
\author{Olav Fønstelien}

\begin{document}
\maketitle

\begin{abstract}
%The abstract gives the reader a quick overview of what has been done and the most important results. Try to be to the point and state your main findings. It could be structured as follows 
% - Short introduction to topic and why its important 
% - Introduce a challenge or unresolved issue with the topic (that you will try to solve) 
% - What have you done to solve this 
% - Main Results 
% - The implications

\end{abstract}

\section{Introduction} \label{sec:intro}
%When you write the introduction you could focus on the following aspects
% - Motivate the reader, the first part of the introduction gives always a motivation and tries to give the overarching ideas
% - What I have done
% - The structure of the report, how it is organized etc

RK4 optimal with regards to computations and error.

%\clearpage
\section{Background: The SIRS Model} \label{sec:background-sirs}
%classical; introduce vaccine; introduce vital dynamics
%present R0=a/b and Rt=as/b; controlling a through repression or s through vaccination
%Vaccination rates and dependency i,f
%ratio needed to be vaccinated based on the f>=c(a/b-1) relationship

The classical \textit{SIRS} model describes the spread of disease in a population of size $N$, with the transitions between the classes of \textit{Susceptible}, $S$; \textit{Infectious}, $I$; and \textit{Recovering}, $R$. The possible transitions of an individual is cyclic;
\begin{equation}
	S \rightarrow I \rightarrow R \rightarrow S,
\end{equation}
and hence the name. The rates of the transition $S \rightarrow I$ is dependent of a transition rate $a$, and the chance $S/N \cdot I/N$ that each of two interacting individuals are in the groups $S$ and $I$;
\begin{equation}
	\mathrm{rate}(S \rightarrow I) = asi,
\end{equation}
where we have let $s,i$ denote the ratios of susceptibles and infectous in the population $N$. Similarly, the rates of transitions from class $I$ to $R$, and $R$ to $S$ are given by the recovery rate $b$ and the immunity loss rate $c$. These processes happen spontaneously and need no second individual, such that
\begin{equation}
\begin{aligned}
	\mathrm{rate}(I \rightarrow R) &= bi \\
	\mathrm{rate}(R \rightarrow S) &= cr
\end{aligned} \quad,
\end{equation}
where again $r$ is the ratio of recovering individual in population $N$.

From this we can set up the coupled differential equations which describe the rates of transitions between the classes in the population:
\begin{equation} \label{eq:sirs-classic}
\begin{aligned}
	s' &= \mathrm{rate}(R \rightarrow S) - \mathrm{rate}(S \rightarrow I) = cr - asi \\
	i' &= \mathrm{rate}(S \rightarrow I) - \mathrm{rate}(I \rightarrow R) = asi - bi \\
	r' &= \mathrm{rate}(I \rightarrow R) - \mathrm{rate}(R \rightarrow S) = bi - cr
\end{aligned} \quad .
\end{equation}

The classical SIRS model reaches steady state when $s'=i'=r'=0$. Observing that $S + I + R = N$, we get the steady-state ratios
\begin{equation} \label{eq:sirs-classic-steady}
\begin{aligned}
	s^* &= b/a \\
	i^* &= \frac{1 - b/a}{1 + b/c} \\
	r^* &= 1 - s^* - i^* \\
\end{aligned} \quad .
\end{equation}
Here we note that for a disease requires that $i^* > 0$ for it to be able to establish itself within a population, meaning that the steady-state ratio of susceptibles $s^* = b/a < 1$. This ratio is otherwise known as the initial reproduction number $R_0$, defined as the rate of reproduction in a perfectly susceptible population -- that is; a population where $s = 1$, except for the initial infected individual. At a later stage, $s < 1$, and the reproduction number is given by $as/b$;
\begin{equation}
\begin{aligned}
	R_0 &= b/a \\
	R_t &= as/b \\
\end{aligned} \quad .
\end{equation}

During the outbreak of disease -- be it the annual influenza or the black plague -- Governments and individuals will try to reduce $R_t$ as far as possible or practicable. This can be done either by suppression, i.e. reducing the transmission rate $a$; or by removing individuals from the $S$ class and letting them enter the $R$ class directly through vaccination. If we denote the rate of vaccination within a population $f$, and limit vaccination to those in the $S$ class only, we can describe the rate transition into $R$ from $S$ as
\begin{equation}
	\mathrm{rate}(S \rightarrow R) = fs.
\end{equation}
Introducing vaccination into Equation (\ref{eq:sirs-classic}) gives us
\begin{equation} \label{eq:sirs-vaccinated}
\begin{aligned}
	s' &= cr - asi - fs \\
	i' &= asi - bi \\
	r' &= bi - cr + fs
\end{aligned} \quad .
\end{equation}
Likewise, the steady-state ratios from Equation (\ref{eq:sirs-classic-steady}) now become
\begin{equation} \label{eq:sirs-vaccinated-steady}
\begin{aligned}
	s^* &= b/a \\
	i^* &= \frac{1 - b/a(1 + f/c)}{1 + b/c} \\
	r^* &= 1 - s^* - i^* \\
\end{aligned} \quad ,
\end{equation}
where we again note that a disease needs $b/a < 1$ in order to establish itself, but given a vaccine, the new steady-state ratio is reduced due to the $f/c$ term in $i^*$.

Given a disease with know transmission and immunity loss rates, we may predict the needed rate of vaccination $f$ in order to remove the disease from the population -- that is; $\mathrm{argmin}_{f} i^*(f) = 0$. From Equation (\ref{eq:sirs-vaccinated-steady}) we see that $f = c(a/b - 1) - i^*a(c/b + 1)$, and $i^* = 0$ is thus achieved when
\begin{equation}
	f \ge c(a/b - 1).
\end{equation}
Likewise, by setting $i = 0$ in Equation (\ref{eq:sirs-vaccinated}) we find large part of the population to vaccinate -- or rather; how large part of the population which has to have antibodies against a pathogen as $i \rightarrow i^* = 0$:
\begin{equation}
	r \rightarrow r^* = 1 - \frac{1}{1 + f/c}.
\end{equation}

At last, we may introduce vital dynamics -- births and deaths -- into the SIRS model to be able to study the effect of deadly diseases over longer periods of time. Given a birth rate $e$, a background death rate $d$ and a the incidence death rate $d_I$, the full form of the SIRS model becomes
\begin{equation} \label{eq:sirs-vitals}
\begin{aligned}
	s' &= cr - asi - fs - ds + e \\
	i' &= asi - bi - di -d_Ii\\
	r' &= bi - cr + fs - dr
\end{aligned} \quad ,
\end{equation}
where we have assumed that newborns are initially susceptible. The steady-state ratios $s^*, i^*, r^*$ are more complicated to attain, and in this report we will find these with the help of the numerical models which we will developed in the following chapter. This will allow us to introduce time-varying rates, such as transmission rates affected by government policy, or delayed availability of vaccines. 

%\clearpage
\section{Methods} \label{sec:methods}
% - Describe the methods and algorithms
% - You need to explain how you implemented the methods and also say something about the structure of your algorithm and present some parts of your code
% - You should plug in some calculations to demonstrate your code, such as selected runs used to validate and verify your results. The latter is extremely important!! A reader needs to understand that your code reproduces selected benchmarks and reproduces previous results, either numerical and/or well-known closed form expressions.

The differential equation for the SIRS model as stated in Equations (\ref{eq:sirs-classic}), (\ref{eq:sirs-vaccinated}), (\ref{eq:sirs-vitals}) does not have an analytical solution. In this chapter we will develop two algorithms for solving them; one based on the Fourth order Runge-Kutta numerical method; and one based on Monte Carlo simulation.

\subsection{Fourth Order Runge-Kutta Algorithm for the SIRS Model} \label{sec:runge-kutta}
The Fourth Order Runge-Kutta method, or simply RK4, is based on \textit{Simpson's Rule} for numerical integration, by which
\begin{equation} \label{eq:simpsons-rule}
	\int_{a}^{b} f(x) dx = \frac{1}{6} \big( f(a) + 4f((b+a)/2) + f(b) \big) + O(h^5),
\end{equation}
-- that is; an approximation of $f$'s slope over $[a,b]$ by the weighted function values at the end points $a,b$ and the midpoint $(a+b)/2$.

Given a first-order differential equation on the form
\begin{equation}
	\frac{dy}{dt} = f(t, y),
\end{equation}
we can approximate $y(t)$ numerically by $y_t$ as
\begin{equation}
\begin{aligned} \label{eq:re4-1st-step}
		y_{i+1} &= y_i + \int_{t_i}^{t_{i+1}} f(t,y) dt \\
		&= \frac{h}{6} \big( f(t_i, y_i) + 4f(t_{i+1/2}, y_{i+1/2}) + f(t_{i+1}, y_{i+1}) \big) + O(h^5)
\end{aligned} \quad ,
\end{equation}
where $h=t_{i+1} - t_i$ is the step size. The midpoint and endpoint values $y_{i+1/2}$ and $y_{i+1}$ are unknown to us. We may approximate $y_{i+1/2}$ as $y^*_{i+1/2}$ by \textit{Euler's Forward Method} with step size $h/2$ such that
\begin{equation}
	y_{i+1/2} = y_i + \frac{h}{2} f(t_i, y_i) + O(h^2) \Rightarrow y^*_{i+1/2} = y_i + \frac{h}{2} f(t_i, y_i),
\end{equation}
but instead of applying this directly into Equation (\ref{eq:re4-1st-step}), RK4 now splits the middle term in two;
\begin{equation}
	4f(t_{i+1/2}, y_{i+1/2}) \rightarrow 2f(t_{i+1/2}, y^{*}_{i+1/2}) + 2f(t_{i+1/2}, y^{**}_{i+1/2}),
\end{equation}
where $y^{**}_{i+1/2}$ in the second term is an improved approximation of $y_{i+1/2}$ using
\begin{equation}
	y^{**}_{i+1/2} = y_i + \frac{h}{2} f(t_{i+1/2}, y^{*}_{i+1/2}).
\end{equation}
Following the same pattern, $y_{i+1}$ in the last term is approximated by $y^*_{i+1}$ using $y^{**}_{i+1/2}$ such that
\begin{equation}
	y^{*}_{i+1} = y_i + h f(t_{i+1/2}, y^{**}_{i+1/2}).
\end{equation}
We end up with an approximation $y^{**}_{i+1}$ which is given by
\begin{equation}
	y^{**}_{i+1} = y_i + \frac{h}{6} \big( f(t_i, y_i) + 2f(t_i + \frac{h}{2}, y^*_{i+1/2}) + 2f(t_i + \frac{h}{2}, y^{**}_{i+1/2}) + f(t_i+h, y^*_{i+1}) \big).
\end{equation}
The error is similar to that of Simpson's Rule, $O(h^5)$, which adds up to $O(h^4)$ over the whole run from $t_1=t_0+h$ to $t_n=t_0+nh$. This comes however at the cost of seven evaluations of $f(t,y)$ at each time step, compared with only one for Euler's method, which again has an overall error $O(h)$.

Algorithmically, the steps are given in Listing \ref{lst:rk4-steps} and is very simple to implement. It runs in $\mathcal{O}(n)$ time and requires no memory of the earlier calculation steps. It is CPU-bound.

\begin{lstlisting}[caption={Fourth order Runge-Kutta algorithm.},label={lst:rk4-steps},escapeinside={@}{@}] [!h]
ALGORITHM RUNGE-KUTTA(f, ti, yi, h)
	Inputs: function f, time ti, value yi at time ti, step size h.
	Outputs: approximation of y at time ti+h.
	
	k1 = h*f(ti, yi)
	k2 = h*f(ti+h/2, yi+k1/2)
	k3 = h*f(ti+h/2, yi+k2/2)
	k4 = h*f(ti+h, yi+k3)
	output yi + (k1 + 2*k2 + 2*k3 + k4)/6

END ALGORITHM
\end{lstlisting}

The SIRS model as described in Equation (\ref{eq:sirs-vitals}) can be implemented directly as it stands using the Runge-Kutta algorithm in Listing \ref{lst:rk4-steps}. We will however add an equation for the population growth rate $n'$, giving the following set of equations;

\begin{equation} \label{eq:sirs-rk4}
\begin{aligned}
	s' &= cr - asi - fs - ds + e \\
	i' &= asi - bi - di -d_Ii \\
	r' &= bi - cr + fs - dr \\
	n' &= e - d - id_I
\end{aligned} \quad .
\end{equation}

We must select a suitable time step $h$, and update all approximations for each iterations, using only the values from the earlier iteration. However, we might update the last equation, say that for $r_i$ based on the results of the others, since $r_i = n_i - s_i - i_i$ at all times. This improves numerical stability. A possible implementation is outlined in Listing \ref{lst:rk4-sirs}.

\begin{lstlisting}[caption={SIRS model Runge-Kutta algorithm.},label={lst:rk4-sirs},escapeinside={@}{@}] [!h]
ALGORITHM SIRS-RUNGE-KUTTA(fs, fi, fn, t0, s0, i0, n0, n, h)
	Inputs: functions fi, fi, fn, initial time t0, 
			initial values s0, i0, n0, time steps n, step size h.
	Outputs: approximation of s, i, r, n at time
			 t=t0+h, t0+2*h, ..., t0+n*h
	
	s = s0
	i = i0
	n = n0
	t = t0
	FOR k IN 1...n DO
		s = RUNGE-KUTTA(fs, t, s, h)
		i = RUNGE-KUTTA(fi, t, i, h)
		n = RUNGE-KUTTA(fn, t, n, h)
		r = n - s - i
		t = t + h
		output s, i, r, n
	END FOR

END ALGORITHM
\end{lstlisting}

Possible extensions to the Fourth Order Runge-Kutta algorithm include the \textit{Runge-Kutta-Fehlberg} algorithm, which applies adaptive time step for error control and optimalization. See \cite{fys4150-notes} for more on this issue as well as numerical methods for solving differential equations in general.

\subsection{Monte Carlo Algorithm for the SIRS Model} \label{sec:monte-carlo}
%lets us calculate discrete values
The Runge-Kutta method inherently implicates continuous values for the ratios of susceptibles $s$, infectious $i$, and recovering $r$ in the population. The Monte Carlo method, on the other hand, lets us calculate the discrete changes in each class, and thereby simulate a possible trajectory for the spread of the disease and the effect it has on the population. 

Before we run the simulation, however, the expectation value at each step in the trajectory is exactly the solution to the differential equation describing the SIRS model;
\begin{equation}
	\mathbb{E}(y_i) = y(t_i)
\end{equation}
The Monte Carlo method will therefore give us an estimate $\bar{\mu}_i$ of the solution if we calculate its mean value. The estimate follows the $t$-distribution and if we run $n$ simulations it has standard deviation $\bar{\sigma}_i$ given by
\begin{equation}
	\bar{\sigma}_i^2 = \frac{1}{n-1} \sum_{j=1}^{n} (y_j - \bar{\mu}_i)^2.
\end{equation}
From this we can calculate a confidence interval for the \textit{true} $\mu_i = y(t_i)$, i.e. the solution to the equations. A $100 \cdot (1-\alpha)$ \% confidence interval is given by
\begin{equation}
	\bar{\mu}_i - t_{\alpha/2, \nu} \frac{\sigma}{\sqrt{n}} < y(t_i) < \bar{\mu}_i + t_{\alpha/2, \nu} \frac{\sigma}{\sqrt{n}},
\end{equation}
where $\nu = n-1$ denotes the $t$-distribution's degrees of freedom. See \cite{devore2012}. As $\nu \rightarrow \infty$, the $t$-distribution converges towards the normal distribution, but for small $n$, i.e in the 10's or 20's, it does matter. A CI$_{90}$ has a multiplier of 1.64 for the normal distribution, while it is 1.83 for the $t$-distribution with $\nu=9$. Corresponding values for CI$_{99}$ are 2.58 and 3.25.

We implement the Monte Carlo method by first discretizing the SIRS model in Equation (\ref{eq:sirs-vitals});
\begin{equation} \label{eq:sirs-vitals-discrete}
\begin{aligned}
	S' &= cR - \frac{aSI}{N} - fS - dS + eN \\
	I' &= \frac{aSI}{N} - bI - dI -d_II\\
	R' &= bI - cR + fS - dR
\end{aligned} \quad ,
\end{equation}
where we have let the upper-case letters $S,I,R,N$ denote discrete variables and their derivatives. Next, we translate the transition rates $\mathrm{rate}(X \rightarrow Y)$ into probabilities that a transition will happen within a time step $\Delta t$ of our simulation;
\begin{equation} \label{eq:transition-probabilities}
\begin{aligned}
	P(S \rightarrow I) &= \frac{aSI}{N} \Delta t \\
	P(S \rightarrow R) &= fS \Delta t \\
	P(S \rightarrow D) &= dS \Delta t \\	
	P(B \rightarrow S) &= eN \Delta t \\	
	P(I \rightarrow R) &= bI \Delta t \\	
	P(I \rightarrow D) &= dI \Delta t \\	
	P(I \rightarrow D_I) &= d_II \Delta t \\	
	P(R \rightarrow S) &= cR \Delta t \\	
	P(R \rightarrow D) &= dR \Delta t \\		
\end{aligned} \quad .
\end{equation}
Here, we have let $D$ denote deaths from other causes than the disease; $D_I$ deaths from the disease; and $B$ denote births, which again are all assumed into class $S$, i.e. susceptible. Since we cannot have any half deaths or half births, and since at any point in the simulation do not know the development in the next steps, we must adapt $\Delta t$ such that at any time at most one individual is likely to transition from one class to another. To find $\Delta t$, observe that for all but the new infections, the highest probability for the transition out of one class $X$ into another class $Y$ occurs when all $N$ individuals are in $X$. For new infections we have the highest rate when exactly half of the population are in each class $S,I$. Consequently, the time step $\Delta t$ must satisfy
\begin{equation}
\begin{aligned}
	\mathrm{argmax}_X  P(X \rightarrow Y) &= qN \Delta t \le 1 \\
	\mathrm{argmax}_{S,I} P(S \rightarrow I) &= \frac{aN}{4} \Delta t \le 1
\end{aligned} \quad .
\end{equation}
For remembering that we may also have time-varying transition rates, such as seasonal vaccination or policy-suppressed transmission rates, we must adapt the time step at each iteration in the simulation such that
\begin{equation} \label{eq:dt-min}
	\Delta t = \frac{1}{N} \cdot \mathrm{min} \bigg( \frac{4}{a}, \frac{1}{b}, \frac{1}{c}, \frac{1}{d}, \frac{1}{d_I}, \frac{1}{e}, \frac{1}{f} \bigg).
\end{equation}

Now, having decided on the time step for iteration $i$, say, we draw a random number $\varepsilon_i \sim \mathcal{U}(0,1)$, and transition one individual from class $X$ to $Y$ if $\varepsilon_i < P_i(X \rightarrow Y)$. Since all transitions \textit{within one iteration} of the simulation are independent, we may reuse $\varepsilon_i$ when we do all the transition tests in that iteration. An outline of the resulting SIRS model Monte Carlo algorithm if given in Listing \ref{lst:mc-sirs}. 

\begin{lstlisting}[caption={SIRS model Monte Carlo algorithm.},label={lst:mc-sirs},escapeinside={@}{@}] [!h]
ALGORITHM SIRS-MONTE-CARLO()
	Inputs: initial population distribution S0, I0, R0, N0,
	time-varying transition rates a, b, c, d, dI, e, f,
	initial time t0, end time tn.
	Outputs: approximation of S, I, R, N at time
	t1=t0+dt1, t2=t0+dt2, ..., tn.
	
	S = S0
	I = I0
	R = R0
	N = N0
	t = t0
	WHILE t <= tn DO
		dt = get_time_step(N, t)  // (@Equation (\ref{eq:dt-min})@)
		epsilon = draw_random_number()
		{collect incremental transitions from 
		 all combinations in @Equation (\ref{eq:transition-probabilities})@.
		 Make all transitions where probability < epsilon}
		
	
		t = t + dt
	END WHILE

END ALGORITHM
\end{lstlisting}


%\clearpage
\section{Results} \label{sec:results}
% - Present your results
% - Give a critical discussion of your work and place it in the correct context.
% - Relate your work to other calculations/studies
% - An eventual reader should be able to reproduce your calculations if she/he wants to do so. All input variables should be properly explained.
% - Make sure that figures and tables should contain enough information in their captions, axis labels etc so that an eventual reader can gain a first impression of your work by studying figures and tables only.

\subsection{The Importance of the Recovery/Transmission Ratio} \label{sec:recovery-rates}
%Show development for various ratios; tie this to R0=a/b and Rt=as/b, also with regards to controlling a through repression or s through vaccination
%present distributions 5<t<10000
%Include seasonal variations


\subsection{Introducing Vital Dynamics} \label{sec:vital-dynamics}
%non-deadly; collapse; equilibrium
%ratio reaches equilibrium independently of the d,dI,e ratio
%Monte Carlo is optimistic -- >99% sure about survival; RK4 more pessimistic -- collapse of the population

\subsection{Introducing Vaccines} \label{sec:vaccines}
%modelling expected i based on f and vice versa
%ratio needed to be vaccinated based on the f>=c(a/b-1) relationship
%seasonal variations in i even if mean of f = fopt

\subsection{Bringing it All Together: Case Study} \label{sec:case-study}
%policies as bathtub curves
%vaccine availability as exponential curve
%effect of early introduction of lockdown vs strictness


%\clearpage
\section{Discussion and Conclusion} \label{sec:conclusion}
% - State your main findings and interpretations
% - Try as far as possible to present perspectives for future work
% - Try to discuss the pros and cons of the methods and possible improvements

%No use for MC in these calculations; more of an academic interest, plus proof of concept
%simulations show that with the early introduction of lockdown has best effect; late strict lockdowns have little effect other than making the summer lulls calmer. Timing is the deciding factor; strictness as modifier of success.


\subsection{Subsection} \label{sec:subsection}

%\clearpage
\bibliographystyle{unsrt}
\bibliography{project5.bib}
\end{document}
