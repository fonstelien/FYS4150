\documentclass[]{article}
\usepackage{amssymb}
\usepackage{amsmath}
\usepackage[utf8]{inputenc}
\usepackage{graphicx}
\usepackage{booktabs}
\usepackage{listings}
\usepackage{color}
\usepackage{tabularx}
\usepackage{hyperref}

\definecolor{dkgreen}{rgb}{0,0.6,0}
\definecolor{gray}{rgb}{0.5,0.5,0.5}
\definecolor{mauve}{rgb}{0.58,0,0.82}

\lstset{frame=tb,
	%language=C++,
	aboveskip=3mm,
	belowskip=3mm,
	showstringspaces=false,
	columns=flexible,
	basicstyle={\small\ttfamily},
	numbers=none,
	numberstyle=\tiny\color{gray},
	keywordstyle=\color{blue},
	commentstyle=\color{dkgreen},
	stringstyle=\color{mauve},
	breaklines=false,
	breakatwhitespace=true,
	tabsize=2
}


\title{FYS4150 H20 - Project 5:\\Disease Modelling with Runge-Kutta and\\Monte Carlo Simulations}
\author{Olav Fønstelien}

\begin{document}
\maketitle

\begin{abstract}
%The abstract gives the reader a quick overview of what has been done and the most important results. Try to be to the point and state your main findings. It could be structured as follows 
% - Short introduction to topic and why its important 
% - Introduce a challenge or unresolved issue with the topic (that you will try to solve) 
% - What have you done to solve this 
% - Main Results 
% - The implications

\end{abstract}

\section{Introduction} \label{sec:intro}
%When you write the introduction you could focus on the following aspects
% - Motivate the reader, the first part of the introduction gives always a motivation and tries to give the overarching ideas
% - What I have done
% - The structure of the report, how it is organized etc


%\clearpage
\section{Background: The SIRS Model} \label{sec:background-sirs}
%classical; introduce vaccine; introduce vital dynamics
%present R0=a/b and Rt=as/b; controlling a through repression or s through vaccination
%Vaccination rates and dependency i,f
%ratio needed to be vaccinated based on the f>=c(a/b-1) relationship

The classical \textit{SIRS} model describes the spread of disease in a population of size $N$, with the transitions between the classes of \textit{Susceptible}, $S$; \textit{Infectious}, $I$; and \textit{Recovering}, $R$. The possible transitions of an individual is cyclic;
\begin{equation}
	S \rightarrow I \rightarrow R \rightarrow S,
\end{equation}
and hence the name. The rates of the transition $S \rightarrow I$ is dependent of a transition rate $a$, and the chance $S/N \cdot I/N$ that each of two interacting individuals are in the groups $S$ and $I$;
\begin{equation}
	\mathrm{rate}(S \rightarrow I) = asi,
\end{equation}
where we have let $s,i$ denote the ratios of susceptibles and infectous in the population $N$. Similarly, the rates of transitions from class $I$ to $R$, and $R$ to $S$ are given by the recovery rate $b$ and the immunity loss rate $c$. These processes happen spontaneously and need no second individual, such that
\begin{equation}
\begin{aligned}
	\mathrm{rate}(I \rightarrow R) &= bi \\
	\mathrm{rate}(R \rightarrow S) &= cr
\end{aligned} \quad,
\end{equation}
where again $r$ is the ratio of recovering individual in population $N$.

From this we can set up the coupled differential equations which describe the rates of transitions between the classes in the population:
\begin{equation} \label{eq:sirs-classic}
\begin{aligned}
	s' &= \mathrm{rate}(R \rightarrow S) - \mathrm{rate}(S \rightarrow I) = cr - asi \\
	i' &= \mathrm{rate}(S \rightarrow I) - \mathrm{rate}(I \rightarrow R) = asi - bi \\
	r' &= \mathrm{rate}(I \rightarrow R) - \mathrm{rate}(R \rightarrow S) = bi - cr
\end{aligned} \quad .
\end{equation}

The classical SIRS model reaches steady state when $s'=i'=r'=0$. Observing that $S + I + R = N$, we get the steady-state ratios
\begin{equation} \label{eq:sirs-classic-steady}
\begin{aligned}
	s^* &= b/a \\
	i^* &= \frac{1 - b/a}{1 + b/c} \\
	r^* &= 1 - s^* - i^* \\
\end{aligned} \quad .
\end{equation}
Here we note that for a disease requires that $i^* > 0$ for it to be able to establish itself within a population, meaning that the steady-state ratio of susceptibles $s^* = b/a < 1$. This ratio is otherwise known as the initial reproduction number $R_0$, defined as the rate of reproduction in a perfectly susceptible population -- that is; a population where $s = 1$, except for the initial infected individual. At a later stage, $s < 1$, and the reproduction number is given by $as/b$;
\begin{equation}
\begin{aligned}
	R_0 &= b/a \\
	R_t &= as/b \\
\end{aligned} \quad .
\end{equation}

During the outbreak of disease -- be it the annual influenza or the black plague -- Governments and individuals will try to reduce $R_t$ as far as possible or practicable. This can be done either by suppression, i.e. reducing the transmission rate $a$; or by removing individuals from the $S$ class and letting them enter the $R$ class directly through vaccination. If we denote the rate of vaccination within a population $f$, and limit vaccination to those in the $S$ class only, we can describe the rate transition into $R$ from $S$ as
\begin{equation}
	\mathrm{rate}(S \rightarrow R) = fs.
\end{equation}
Introducing vaccination into Equation (\ref{eq:sirs-classic}) gives us
\begin{equation} \label{eq:sirs-vaccinated}
\begin{aligned}
	s' &= cr - asi - fs \\
	i' &= asi - bi \\
	r' &= bi - cr + fs
\end{aligned} \quad .
\end{equation}
Likewise, the steady-state ratios from Equation (\ref{eq:sirs-classic-steady}) now become
\begin{equation} \label{eq:sirs-vaccinated-steady}
\begin{aligned}
	s^* &= b/a \\
	i^* &= \frac{1 - b/a(1 + f/c)}{1 + b/c} \\
	r^* &= 1 - s^* - i^* \\
\end{aligned} \quad ,
\end{equation}
where we again note that a disease needs $b/a < 1$ in order to establish itself, but given a vaccine, the new steady-state ratio is reduced due to the $f/c$ term in $i^*$.

Given a disease with know transmission and immunity loss rates, we may predict the needed rate of vaccination $f$ in order to remove the disease from the population -- that is; $\mathrm{argmin}_{f} i^*(f) = 0$. From Equation (\ref{eq:sirs-vaccinated-steady}) we see that $f = c(a/b - 1) - i^*a(c/b + 1)$, and $i^* = 0$ is thus achieved when
\begin{equation}
	f \ge c(a/b - 1).
\end{equation}
Likewise, by setting $i = 0$ in Equation (\ref{eq:sirs-vaccinated}) we find large part of the population to vaccinate -- or rather; how large part of the population which has to have antibodies against a pathogen as $i \rightarrow i^* = 0$:
\begin{equation}
	r \rightarrow r^* = 1 - \frac{1}{1 + f/c}.
\end{equation}

At last, we may introduce vital dynamics -- births and deaths -- into the SIRS model to be able to study the effect of deadly diseases over longer periods of time. Given a birth rate $e$, a background death rate $d$ and a the incidence death rate $d_I$, the full form of the SIRS model becomes
\begin{equation} \label{eq:sirs-vitals}
\begin{aligned}
	s' &= cr - asi - fs - ds + e \\
	i' &= asi - bi - di -d_Ii\\
	r' &= bi - cr + fs - dr
\end{aligned} \quad ,
\end{equation}
where we have assumed that newborns are initially susceptible. The steady-state ratios $s^*, i^*, r^*$ are more complicated to attain, and in this report we will find these with the help of the numerical models which we will developed in the following chapter. This will allow us to introduce time-varying rates, such as transmission rates affected by government policy, or delayed availability of vaccines. 

%\clearpage
\section{Methods} \label{sec:methods}
% - Describe the methods and algorithms
% - You need to explain how you implemented the methods and also say something about the structure of your algorithm and present some parts of your code
% - You should plug in some calculations to demonstrate your code, such as selected runs used to validate and verify your results. The latter is extremely important!! A reader needs to understand that your code reproduces selected benchmarks and reproduces previous results, either numerical and/or well-known closed form expressions.

The differential equation for the SIRS model as stated in Equations (\ref{eq:sirs-classic}), (\ref{eq:sirs-vaccinated}), (\ref{eq:sirs-vitals}) does not have an analytical solution. In this chapter we will develop two algorithms for solving them; one based on the Fourth order Runge-Kutta numerical method; and one based on Monte Carlo simulation.

\subsection{Fourth Order Runge-Kutta Algorithm for the SIRS Model} \label{sec:runge-kutta}
The Fourth Order Runge-Kutta method, or simply RK4, is based on \textit{Simpson's rule} for numerical integration, by which
\begin{equation}
	\int_{a}^{b} f(x) dx = \frac{1}{6} \big( f(a) + 4f((b+a)/2) + f(b) \big) + O(h^5).
\end{equation}
Given a first-order differential equation on the form
\begin{equation}
	\frac{dy}{dt} = f(t, y),
\end{equation}
we can approximate $y(t)$ numerically by $y_t$ as
\begin{equation}
\begin{aligned} \label{eq:re4-1st-step}
		y_{i+1} &= y_i + \int_{t_i}^{t_{i+1}} f(t,y) dt \\
		&= \frac{1}{6} \big( f(t_i, y_i) + 4f(t_{i+1/2}, y_{i+1/2}) + f(t_{i+1}, y_{i+1}) \big) + O(h^5)
\end{aligned} \quad .
\end{equation}
The midpoint value $y_{i+1/2}$ and $y_{i+1}$ are unknown to us. We may approximate $y_{i+1/2}$ as $y^*_{i+1/2}$ by \textit{Euler's Forward Method} with step size $h/2$ such that
\begin{equation}
	y_{i+1/2} = y_i + \frac{h}{2} f(t_i, y_i) + O(h^2) \Rightarrow y^*_{i+1/2} = y_i + \frac{h}{2} f(t_i, y_i),
\end{equation}
but instead of applying this directly into Equation (\ref{eq:re4-1st-step}), RK4 now splits the middle term in two;
\begin{equation}
	4f(t_{i+1/2}, y_{i+1/2}) \rightarrow 2f(t_{i+1/2}, y^{*}_{i+1/2}) + 2f(t_{i+1/2}, y^{**}_{i+1/2}),
\end{equation}
where $y^{**}_{i+1/2}$ in the second term is an improved approximation of $y_{i+1/2}$ using
\begin{equation}
	y^{**}_{i+1/2} = y_i + \frac{h}{2} f(t_i, y^*_i).
\end{equation}
Following the same pattern, $y_{i+1}$ in the last term is approximated by $y^*_{i+1}$ using $y^{**}_{i+1/2}$ such that
\begin{equation}
	y^{*}_{i+1} = y_i + \frac{h}{2} f(t_{i+1/2}, y^{**}_{i+1/2}).
\end{equation}



\subsection{Monte Carlo Algorithm for the SIRS Model} \label{sec:monte-carlo}
%lets us calculate discrete values


%\clearpage
\section{Results} \label{sec:results}
% - Present your results
% - Give a critical discussion of your work and place it in the correct context.
% - Relate your work to other calculations/studies
% - An eventual reader should be able to reproduce your calculations if she/he wants to do so. All input variables should be properly explained.
% - Make sure that figures and tables should contain enough information in their captions, axis labels etc so that an eventual reader can gain a first impression of your work by studying figures and tables only.

\subsection{The Importance of the Recovery/Transmission Ratio} \label{sec:recovery-rates}
%Show development for various ratios; tie this to R0=a/b and Rt=as/b, also with regards to controlling a through repression or s through vaccination
%present distributions 5<t<10000
%Include seasonal variations


\subsection{Introducing Vital Dynamics} \label{sec:vital-dynamics}
%non-deadly; collapse; equilibrium
%ratio reaches equilibrium independently of the d,dI,e ratio
%Monte Carlo is optimistic -- >99% sure about survival; RK4 more pessimistic -- collapse of the population

\subsection{Introducing Vaccines} \label{sec:vaccines}
%modelling expected i based on f and vice versa
%ratio needed to be vaccinated based on the f>=c(a/b-1) relationship
%seasonal variations in i even if mean of f = fopt

\subsection{Bringing it All Together: Case Study} \label{sec:case-study}
%policies as bathtub curves
%vaccine availability as exponential curve
%effect of early introduction of lockdown vs strictness


%\clearpage
\section{Discussion and Conclusion} \label{sec:conclusion}
% - State your main findings and interpretations
% - Try as far as possible to present perspectives for future work
% - Try to discuss the pros and cons of the methods and possible improvements

%No use for MC in these calculations; more of an academic interest, plus proof of concept
%simulations show that with the early introduction of lockdown has best effect; late strict lockdowns have little effect other than making the summer lulls calmer. Timing is the deciding factor; strictness as modifier of success.


\subsection{Subsection} \label{sec:subsection}

%\clearpage
\bibliographystyle{unsrt}
\bibliography{project5.bib}
\end{document}
