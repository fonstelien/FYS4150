\documentclass[]{article}
\usepackage{amssymb}
\usepackage{amsmath}
\usepackage[utf8]{inputenc}
\usepackage{graphicx}
\usepackage{booktabs}
\usepackage{listings}
\usepackage{color}
\usepackage{tabularx}
\usepackage{hyperref}

\definecolor{dkgreen}{rgb}{0,0.6,0}
\definecolor{gray}{rgb}{0.5,0.5,0.5}
\definecolor{mauve}{rgb}{0.58,0,0.82}

\lstset{frame=tb,
	language=C++,
	aboveskip=3mm,
	belowskip=3mm,
	showstringspaces=false,
	columns=flexible,
	basicstyle={\small\ttfamily},
	numbers=none,
	numberstyle=\tiny\color{gray},
	keywordstyle=\color{blue},
	commentstyle=\color{dkgreen},
	stringstyle=\color{mauve},
	breaklines=false,
	breakatwhitespace=true,
	tabsize=2
}

\title{FYS4150 H20 Project 1}
\author{Olav Fønstelien}

\begin{document}
\maketitle

\begin{abstract}
%The abstract gives the reader a quick overview of what has been done and the most important results. Try to be to the point and state your main findings.

\end{abstract}

\section{Introduction}
%When you write the introduction you could focus on the following aspects
%-Motivate the reader, the first part of the introduction gives always a motivation and tries to give the overarching ideas
%-What I have done
%-The structure of the report, how it is organized etc

We will study the numerical solution of eigenvalue problems for second order differential equations 
\[
\frac{d^2 u(\rho)}{d\rho^2} = -\lambda u(\rho),
\]
and
\[
-\frac{d^2}{d\rho^2} u(\rho) + \rho^2u(\rho)  = \lambda u(\rho).
\]
We will apply Dirichlet boundary conditions and.

This work is my submission for the second project in the course FYS4150 given at University of Oslo, autumn 2020 \cite{fys4150},\cite{fys4150-p2}. I have implemented the algorithms in C++ using the Armadillo library. You will find the source code at my repository \url{https://github.com/fonstelien/FYS4150/tree/master/project2}.


\section{Methods}
%-Describe the methods and algorithms
%-You need to explain how you implemented the methods and also say something about the structure of your algorithm and present some parts of your code
%-You should plug in some calculations to demonstrate your code, such as selected runs used to validate and verify your results. The latter is extremely important!! A reader needs to understand that your code reproduces selected benchmarks and reproduces previous results, either numerical and/or well-known closed form expressions.

\subsection{Jacobi eigenvalue algorithm}
Given an eigenvalue problem \[\] The Jacobi algorithm is based on the iterative multiplication of 


\subsection{Polynomial expansion method}


\section{Results}
%-Present your results
%-Give a critical discussion of your work and place it in the correct context.
%-Relate your work to other calculations/studies
%-An eventual reader should be able to reproduce your calculations if she/he wants to do so. All input variables should be properly explained.
%-Make sure that figures and tables should contain enough information in their captions, axis labels etc so that an eventual reader can gain a first impression of your work by studying figures and tables only.


\section{Conclusion}
%-State your main findings and interpretations
%-Try as far as possible to present perspectives for future work
%-Try to discuss the pros and cons of the methods and possible improvements




\bibliographystyle{plain}
\bibliography{project1.bib}

\end{document}
